%%%%%%%%%%%%%%%%%%%%%%%%%%%%%%%%%%%%%%%%%
% Beamer Presentation
% LaTeX Template
% Version 1.0 (10/11/12)
%
% This template has been downloaded from:
% http://www.LaTeXTemplates.com
%
% License:
% CC BY-NC-SA 3.0 (http://creativecommons.org/licenses/by-nc-sa/3.0/)
%
%%%%%%%%%%%%%%%%%%%%%%%%%%%%%%%%%%%%%%%%%

%----------------------------------------------------------------------------------------
%    PACKAGES AND THEMES
%----------------------------------------------------------------------------------------

\documentclass{beamer}

\usepackage[utf8]{inputenc}
\usepackage[T1]{fontenc}
\usepackage{listings}
\usepackage{color}
\usepackage{subcaption}

\mode<presentation> {

% The Beamer class comes with a number of default slide themes
% which change the colors and layouts of slides. Below this is a list
% of all the themes, uncomment each in turn to see what they look like.

%\usetheme{default}
%\usetheme{AnnArbor}
%\usetheme{Antibes}
%\usetheme{Bergen}
%\usetheme{Berkeley}
%\usetheme{Berlin}
\usetheme{Boadilla}  % good
%\usetheme{CambridgeUS}
%\usetheme{Copenhagen}
%\usetheme{Darmstadt}
%\usetheme{Dresden}
%\usetheme{Frankfurt}
%\usetheme{Goettingen}
%\usetheme{Hannover}
%\usetheme{Ilmenau}
%\usetheme{JuanLesPins}
%\usetheme{Luebeck}
%\usetheme{Madrid}
%\usetheme{Malmoe}
%\usetheme{Marburg}
%\usetheme{Montpellier}
%\usetheme{PaloAlto}
%\usetheme{Pittsburgh}
%\usetheme{Rochester}
%\usetheme{Singapore}
%\usetheme{Szeged}
%\usetheme{Warsaw}

% As well as themes, the Beamer class has a number of color themes
% for any slide theme. Uncomment each of these in turn to see how it
% changes the colors of your current slide theme.

%\usecolortheme{albatross}
\usecolortheme{beaver}  % simple red
%\usecolortheme{beetle}
%\usecolortheme{crane}  % yellow
%\usecolortheme{dolphin}
%\usecolortheme{dove}
%\usecolortheme{fly}
%\usecolortheme{lily}
%\usecolortheme{orchid}
%\usecolortheme{rose}
%\usecolortheme{seagull}
%\usecolortheme{seahorse}  % orig
%\usecolortheme{whale}
%\usecolortheme{wolverine}  % old yellow wolverine

%\setbeamertemplate{footline} % To remove the footer line in all slides uncomment this line
%\setbeamertemplate{footline}[page number] % To replace the footer line in all slides with a simple slide count uncomment this line

%\setbeamertemplate{navigation symbols}{} % To remove the navigation symbols from the bottom of all slides uncomment this line
}

\usepackage{graphicx} % Allows including images
\usepackage{booktabs} % Allows the use of \toprule, \midrule and \bottomrule in tables

%----------------------------------------------------------------------------------------
%    TITLE PAGE
%----------------------------------------------------------------------------------------

\title[Exposing resource limits to containers with LXCFS]{} % The short title appears at the bottom of every slide, the full title is only on the title page

\author{Serge Hallyn} % Your name
\institute{Cisco}
\date{\today} % Date, can be changed to a custom date

\begin{document}

\begin{frame}
\titlepage % Print the title page as the first slide
\end{frame}

\begin{frame}[fragile]
\frametitle{Problem}
	\begin{itemize}
	\item Cgroups impose resource limits on services
	\item Services query resource availability through /proc
	\item Then scale per the reported availability
	\item /proc is system-level
\pause
	\item Solutions:
		\begin{itemize}
		\item Virtualize /proc etc in kernel
		\item Update user-space to consider cgroup limits etc
		\item Libresource (http://github.com/lxc/libresource)
		\item LXCFS (http://github.com/hallyn/lxcfs)
		\end{itemize}
	\end{itemize}
\end{frame}

\begin{frame}[fragile]
\frametitle{LXFS}
	\begin{itemize}
	\item Overmounts certain /proc files:
		\begin{itemize}
		\item /proc/cpuinfo
		\item /proc/diskstats
		\item /proc/meminfo
		\item /proc/stat
		\item /proc/swaps
		\item /proc/uptime
		\end{itemize}
	\item Used to virtualize cgroupfs (before cgroup namespaces)
\pause
	\item Unimplemented features:
		\begin{itemize}
		\item btime
		\item loadavg
		\item /sys/devices/system/cpu \
Open question: If container has cpus 0 and 3, should we show
cpu0 and cpu1, or cpu0 and cpu3?
		\item sysinfo (out of scope)
		\end{itemize}
	\end{itemize}
\end{frame}

\begin{frame}
\frametitle{Questions/Comments?}
\begin{itemize}
\item Join! :)
\item shallyn@cisco.com
\item serge@hallyn.com
\end{itemize}
\end{frame}

%------------------------------------------------

\end{document} 
