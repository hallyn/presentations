%%%%%%%%%%%%%%%%%%%%%%%%%%%%%%%%%%%%%%%%%
% Beamer Presentation
% LaTeX Template
% Version 1.0 (10/11/12)
%
% This template has been downloaded from:
% http://www.LaTeXTemplates.com
%
% License:
% CC BY-NC-SA 3.0 (http://creativecommons.org/licenses/by-nc-sa/3.0/)
%
%%%%%%%%%%%%%%%%%%%%%%%%%%%%%%%%%%%%%%%%%

%----------------------------------------------------------------------------------------
%    PACKAGES AND THEMES
%----------------------------------------------------------------------------------------

\documentclass{beamer}

\usepackage[utf8]{inputenc}
\usepackage[T1]{fontenc}
\usepackage{listings}
\usepackage{color}

\mode<presentation> {

% The Beamer class comes with a number of default slide themes
% which change the colors and layouts of slides. Below this is a list
% of all the themes, uncomment each in turn to see what they look like.

%\usetheme{default}
%\usetheme{AnnArbor}
%\usetheme{Antibes}
%\usetheme{Bergen}
%\usetheme{Berkeley}
%\usetheme{Berlin}
\usetheme{Boadilla}  % good
%\usetheme{CambridgeUS}
%\usetheme{Copenhagen}
%\usetheme{Darmstadt}
%\usetheme{Dresden}
%\usetheme{Frankfurt}
%\usetheme{Goettingen}
%\usetheme{Hannover}
%\usetheme{Ilmenau}
%\usetheme{JuanLesPins}
%\usetheme{Luebeck}
%\usetheme{Madrid}
%\usetheme{Malmoe}
%\usetheme{Marburg}
%\usetheme{Montpellier}
%\usetheme{PaloAlto}
%\usetheme{Pittsburgh}
%\usetheme{Rochester}
%\usetheme{Singapore}
%\usetheme{Szeged}
%\usetheme{Warsaw}

% As well as themes, the Beamer class has a number of color themes
% for any slide theme. Uncomment each of these in turn to see how it
% changes the colors of your current slide theme.

%\usecolortheme{albatross}
\usecolortheme{beaver}  % simple red
%\usecolortheme{beetle}
%\usecolortheme{crane}  % yellow
%\usecolortheme{dolphin}
%\usecolortheme{dove}
%\usecolortheme{fly}
%\usecolortheme{lily}
%\usecolortheme{orchid}
%\usecolortheme{rose}
%\usecolortheme{seagull}
%\usecolortheme{seahorse}  % orig
%\usecolortheme{whale}
%\usecolortheme{wolverine}  % old yellow wolverine

%\setbeamertemplate{footline} % To remove the footer line in all slides uncomment this line
%\setbeamertemplate{footline}[page number] % To replace the footer line in all slides with a simple slide count uncomment this line

%\setbeamertemplate{navigation symbols}{} % To remove the navigation symbols from the bottom of all slides uncomment this line
}

\usepackage{graphicx} % Allows including images
\usepackage{booktabs} % Allows the use of \toprule, \midrule and \bottomrule in tables

%----------------------------------------------------------------------------------------
%    TITLE PAGE
%----------------------------------------------------------------------------------------

\title[Unprivileged containers]{Completely unprivileged containers} % The short title appears at the bottom of every slide, the full title is only on the title page

\author{Serge Hallyn} % Your name
\institute{LXC project}
\date{\today} % Date, can be changed to a custom date

\begin{document}

\begin{frame}
\titlepage % Print the title page as the first slide
\end{frame}

\begin{frame}[fragile]
\frametitle{Virtualization}
	\begin{itemize}
	\item Making virtual instances of hardware, OS, or functions
		\begin{itemize}
		\item paravirt
		\item hardware emulation
		\item OS emulation
		\item NFV, \ldots
		\end{itemize}
\pause
	\item Motivations
		\begin{itemize}
		\item Hide architecture differences
		\item Support multi-user
		\item Test "next level down" (i.e. new kernel)
		\item Scale
		\end{itemize}
	\end{itemize}
\end{frame}

\begin{frame}[fragile]
\frametitle{Virtualization}
	\begin{itemize}
	\item Virtual machines (traditionally)
		\begin{itemize}
		\item May emulate or para-virt hardware
		\item Run an operating system (Linux, Windows, \ldots)
		\item Accessing host resources is like intra-host access
		\item Sharing finagled in after the fact
		\item Para-virt: One machine appearing a $> 1$
		\end{itemize}
\pause
	\item Containers
		\begin{itemize}
		\item OS Level Virtualization
		\item Virtualize OS resources at syscall(2) level
		\item One OS appearing as $> 1$
		\end{itemize}
	\end{itemize}
\end{frame}

\begin{frame}[fragile]
\frametitle{Virtualization}
	\begin{itemize}
	\item Pre-linux containers were like light vms
		\begin{itemize}
		\item Jails
		\item Solaris zones
		\item Linux-vserver
		\item Openvz
		\end{itemize}
	\item Focus on 
		\begin{itemize}
		\item performance (no emulation or OS overhead)
		\item isolation
		\item not sharing
		\end{itemize}
	\end{itemize}
\end{frame}

\begin{frame}[fragile]
\frametitle{Linux Containers}
	\begin{itemize}
	\item Do not exist
		\begin{itemize}
		\item Cgroups
		\item Namespaces
		\end{itemize}
	\item Tools written to hide this
		\begin{itemize}
		\item \verb@docker run --rm -it ubuntu bash@
		\item \verb@lxc launch ubuntu:xenial x1@
		\end{itemize}
	\end{itemize}
\end{frame}

\begin{frame}
\frametitle{Using namespaces by hand}
\begin{center}
\begin{figure}
\lstinputlisting{lxcunshare.sh}
\end{figure}
\pause
\vspace{1in}
\item Why sudo?
\end{center}

	\begin{itemize}
	\item Insert setuid description of figure here
	\end{itemize}
\end{frame}

\begin{frame}
\frametitle{Uid namespace}
	\begin{itemize}
	\item By default, uid map is $\left\{0:2^{32}-1\right\} \rightarrow \left\{ 0:2^{31}-1\right\}$
	\item New uid namespace has no mapping ($\varnothing$)
	\item Unprivileged user may map own uid to any namespace id
	\item Namespace id 0 is privileged over namespace's resources and may unshare all namespaces
	\item Setuid-root programs delegate subuids
		\begin{itemize}
		\item newuidmap using /etc/subuid
		\item newgidmap using /etc/subgid
		\end{itemize}
	\end{itemize}

\end{frame}

\begin{frame}
\frametitle{Putting it together}
Let's make a rootfs
\end{frame}

\begin{frame}
\frametitle{Networking}
	\begin{itemize}
	\item User namespace can unshare network namespace \\
		which it then owns
	\item User ns cannot ``hook into'' host namespace
	\item Solution: delegate bridges
		\begin{itemize}
		\item Be careful! nics can spoof each other
		\item /etc/lxc/lxc-usernet: user veth bridge number
		\end{itemize}
	\item lxc-user-nic
		\begin{itemize}
		\item Creates veth pair
		\item Inserts one into container
		\item Hooks other into specified host bridge (if permitted)
		\end{itemize}
	\end{itemize}
\end{frame}

\begin{frame}
\frametitle{Questions/Comments?}
\begin{itemize}
\item {\url http://linuxcontainers.org}
\item {\tt lxc-\{users,devel\}@lists.linuxcontainers.org}
\item serge@hallyn.com
\end{itemize}
\end{frame}

%------------------------------------------------

\end{document} 
