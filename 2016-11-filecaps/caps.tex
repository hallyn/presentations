%%%%%%%%%%%%%%%%%%%%%%%%%%%%%%%%%%%%%%%%%
% Beamer Presentation
% LaTeX Template
% Version 1.0 (10/11/12)
%
% This template has been downloaded from:
% http://www.LaTeXTemplates.com
%
% License:
% CC BY-NC-SA 3.0 (http://creativecommons.org/licenses/by-nc-sa/3.0/)
%
%%%%%%%%%%%%%%%%%%%%%%%%%%%%%%%%%%%%%%%%%

%----------------------------------------------------------------------------------------
%    PACKAGES AND THEMES
%----------------------------------------------------------------------------------------

\documentclass{beamer}

\usepackage[utf8]{inputenc}
\usepackage[T1]{fontenc}
\usepackage{listings}
\usepackage{color}
\usepackage{subcaption}

\mode<presentation> {

% The Beamer class comes with a number of default slide themes
% which change the colors and layouts of slides. Below this is a list
% of all the themes, uncomment each in turn to see what they look like.

%\usetheme{default}
%\usetheme{AnnArbor}
%\usetheme{Antibes}
%\usetheme{Bergen}
%\usetheme{Berkeley}
%\usetheme{Berlin}
\usetheme{Boadilla}  % good
%\usetheme{CambridgeUS}
%\usetheme{Copenhagen}
%\usetheme{Darmstadt}
%\usetheme{Dresden}
%\usetheme{Frankfurt}
%\usetheme{Goettingen}
%\usetheme{Hannover}
%\usetheme{Ilmenau}
%\usetheme{JuanLesPins}
%\usetheme{Luebeck}
%\usetheme{Madrid}
%\usetheme{Malmoe}
%\usetheme{Marburg}
%\usetheme{Montpellier}
%\usetheme{PaloAlto}
%\usetheme{Pittsburgh}
%\usetheme{Rochester}
%\usetheme{Singapore}
%\usetheme{Szeged}
%\usetheme{Warsaw}

% As well as themes, the Beamer class has a number of color themes
% for any slide theme. Uncomment each of these in turn to see how it
% changes the colors of your current slide theme.

%\usecolortheme{albatross}
\usecolortheme{beaver}  % simple red
%\usecolortheme{beetle}
%\usecolortheme{crane}  % yellow
%\usecolortheme{dolphin}
%\usecolortheme{dove}
%\usecolortheme{fly}
%\usecolortheme{lily}
%\usecolortheme{orchid}
%\usecolortheme{rose}
%\usecolortheme{seagull}
%\usecolortheme{seahorse}  % orig
%\usecolortheme{whale}
%\usecolortheme{wolverine}  % old yellow wolverine

%\setbeamertemplate{footline} % To remove the footer line in all slides uncomment this line
%\setbeamertemplate{footline}[page number] % To replace the footer line in all slides with a simple slide count uncomment this line

%\setbeamertemplate{navigation symbols}{} % To remove the navigation symbols from the bottom of all slides uncomment this line
}

\usepackage{graphicx} % Allows including images
\usepackage{booktabs} % Allows the use of \toprule, \midrule and \bottomrule in tables

%----------------------------------------------------------------------------------------
%    TITLE PAGE
%----------------------------------------------------------------------------------------

\title[Targeted File Capabilities]{Making File Capabilities Work in Containers} % The short title appears at the bottom of every slide, the full title is only on the title page

\author{Serge Hallyn} % Your name
\institute{LXC project}
\date{\today} % Date, can be changed to a custom date

\begin{document}

\begin{frame}
\titlepage % Print the title page as the first slide
\end{frame}

\begin{frame}[fragile]
\frametitle{Privilege}
	\begin{itemize}
	\item Uid 0
		\begin{itemize}
		\item Owns host files
		\item Has implicit privileges (usually)
		\end{itemize}
\pause
	\item Posix Capabilities
		\begin{itemize}
		\item Pieces of old "root powers"
		\item Disembodied from uid 0
		\item File capabilities can be forced
		\end{itemize}
	\end{itemize}
\end{frame}

\begin{frame}[fragile]
\frametitle{Two ways to start unprivileged}
	\begin{itemize}
	\item Start as root (or setuid), drop caps
	\item File capabilities
		\begin{itemize}
		\item setcap cap\_sys\_admin=pe helloworld
		\end{itemize}
	\end{itemize}
\end{frame}

\begin{frame}
\frametitle{UID Namespaces}
	\begin{itemize}
	\item Map uids
	\item Setuid-root is safe to delegate
	\item File capabilities have no ties to uid
	\end{itemize}
\end{frame}

\begin{frame}
\frametitle{Solution}
	\begin{itemize}
	\item New, targeted file capabilities version
	\item Includes kuid of ns root id
	\item Filecaps are only used if targeted kuid owns current userns
	\item Implicitly rewrite caps in user namespace
	\item One capability per file
	\item May overwrite if privileged over cap target
	\end{itemize}
\end{frame}


\begin{frame}
\frametitle{Questions/Comments?}
\begin{itemize}
\item {\url http://linuxcontainers.org}
\item {\tt lxc-\{users,devel\}@lists.linuxcontainers.org}
\item serge@hallyn.com
\end{itemize}
\end{frame}

%------------------------------------------------

\end{document} 
